%%%%%%%%%%%%%%%%%%%%%%%%%%%%%%%%%%%%%%%%%
% Beamer Presentation
% LaTeX Template
% Version 1.0 (10/11/12)
%
% This template has been downloaded from:
% http://www.LaTeXTemplates.com
%
% License:
% CC BY-NC-SA 3.0 (http://creativecommons.org/licenses/by-nc-sa/3.0/)
%
%%%%%%%%%%%%%%%%%%%%%%%%%%%%%%%%%%%%%%%%%

%----------------------------------------------------------------------------------------
%	PACKAGES AND THEMES
%----------------------------------------------------------------------------------------

\documentclass{beamer}

\mode<presentation> {

% The Beamer class comes with a number of default slide themes
% which change the colors and layouts of slides. Below this is a list
% of all the themes, uncomment each in turn to see what they look like.

%\usetheme{default}
%\usetheme{AnnArbor}
%\usetheme{Antibes}
%\usetheme{Bergen}
%\usetheme{Berkeley}
%\usetheme{Berlin}
%\usetheme{Boadilla}
%\usetheme{CambridgeUS}
%\usetheme{Copenhagen}
%\usetheme{Darmstadt}
%\usetheme{Dresden}
%\usetheme{Frankfurt}
\usetheme{Goettingen}
%\usetheme{Hannover}
%\usetheme{Ilmenau}
%\usetheme{JuanLesPins}
%\usetheme{Luebeck}
%\usetheme{Madrid}
%\usetheme{Malmoe}
%\usetheme{Marburg}
%\usetheme{Montpellier}
%\usetheme{PaloAlto}
%\usetheme{Pittsburgh}
%\usetheme{Rochester}
%\usetheme{Singapore}
%\usetheme{Szeged}
%\usetheme{Warsaw}

% As well as themes, the Beamer class has a number of color themes
% for any slide theme. Uncomment each of these in turn to see how it
% changes the colors of your current slide theme.

%\usecolortheme{albatross}
%\usecolortheme{beaver}
%\usecolortheme{beetle}
%\usecolortheme{crane}
\usecolortheme{dolphin}
%\usecolortheme{dove}
%\usecolortheme{fly}
%\usecolortheme{lily}
%\usecolortheme{orchid}
%\usecolortheme{rose}
%\usecolortheme{seagull}
%\usecolortheme{seahorse}
%\usecolortheme{whale}
%\usecolortheme{wolverine}

%\setbeamertemplate{footline} % To remove the footer line in all slides uncomment this line
%\setbeamertemplate{footline}[page number] % To replace the footer line in all slides with a simple slide count uncomment this line

%\setbeamertemplate{navigation symbols}{} % To remove the navigation symbols from the bottom of all slides uncomment this line
}

\usepackage{graphicx} % Allows including images
\usepackage{booktabs} % Allows the use of \toprule, \midrule and \bottomrule in tables
\usepackage{hyperref}
\usepackage[utf8]{inputenc}
\usepackage{rotating}
\usepackage[export]{adjustbox}

%----------------------------------------------------------------------------------------
%	TITLE PAGE
%----------------------------------------------------------------------------------------

\title[Lehrvortrag. Universität Potsdam.\\2019]{\LARGE Fragmentierung und Habitatverlust\\[0.1cm]
\includegraphics[width=0.4\textwidth]{Figures/hab_loss_frag.jpg}\vspace{-0.4cm}} % The short title appears at the bottom of every slide, the full title is only on the title page

\author[Avit K. Bhowmik]{\vspace{-0.4cm}\textbf{Dr. Avit K. Bhowmik}} % Your name
\institute[The Royal Swedish Academy of Sciences] % Your institution as it will appear on the bottom of every slide, may be shorthand to save space
{\vspace{-0.3cm}
\textit{avit.bhowmik@futureearth.org}% Your email address
}
\date[\small January 25, 2019]{January 25, 2019 \\ \raggedright
\includegraphics[width=0.3\textwidth]{Figures/KVA_logo.png}} % Date, can be changed to a custom date

\begin{document}

\begin{frame}
\titlepage % Print the title page as the first slide
\end{frame}

%------------------------------------------------

\begin{frame}
\frametitle{Über mich}
\begin{columns}
\column{.6\textwidth}
\begin{itemize}
\item Postdoc at Future Earth, Royal Swedish Academy of Sciences
\item Bachelor in Stadt- und Regionalplanung, Master in Geospatial Technologies, Promotion in räumlicher Ökologie
\item Lehrerfahrung: Räumliche Ökologie und Umweltgesundheit, Geoinformatik, Geostatistik und R, Resilienz und Nachhaltigkeit
\end{itemize}
\hspace{0.5cm}
\column{.4\textwidth}
\raggedleft
\includegraphics[width=\textwidth]{Figures/pro.png}\\
\centering
\tiny Lehrvortrag im ``Global Solutions Program, Google''
\end{columns}
\end{frame}

%-----------------------------------------------

\begin{frame}
  \frametitle{Kontext des Lehrvotrages}
\begin{itemize}
\item Studiengang: B.Sc. Geoökologie,\\Module - Geoökologie I \& II
\item Potenzieller Kurs: Grundlegende Konzepte in Geoökologie
\item \alert{Potenzielle Kursinhalte}:
\begin{itemize}
\item Einführung: Ursachen für den Biodiversitätsverlust
\item Konzepte: Habitat, Habitatverlust und Fragmentierung
\item Ursachen und Auswirkungen des Habitatverlusts
\item Maßnamen zur Erhaltung der Biodiversität: Deutschland und Europa
\end{itemize}
\item Inhalt der Vorlesung kann online unter dem folgenden Link abgerufen werden: \alert{https://github.com/AvitBhowmik/testUP} (siehe auch Handout)
\end{itemize}
\end{frame}

%------------------------------------------------

\begin{frame}
  \frametitle{Erwartungen}
\begin{itemize}
\item \alert{Eure Erwartungen:} Umfrage vor Kursbeginn, z.B. Google Forms
\item \alert{Meine Erwartungen:}
\begin{itemize}
\item Rückfragen und kritisches Denken
\item “Problembasiertes Lernen”
\end{itemize}
\end{itemize}
\raggedleft
\includegraphics[width=0.5\textwidth]{Figures/Questions.png}
\end{frame}

%------------------------------------------------

\begin{frame}
  \frametitle{Lernziele und Wiederholungsfragen}
\begin{itemize}
\item Verständnis der Konzepte: Habitat, Habitatverlust und Fragmentierung
	\begin{itemize}
	\item Welche Rolle spielt das Habitat für Arten?
	\item Durch welche Prozesse entsteht ein Habitatverlust oder Framgmentierung?
	\end{itemize}
\item Wissen über die Ursachen und Auswirkungen des Habitatverlusts
	\begin{itemize}
	\item Was sind die zentralen Ursachen von Habitatverlust?
	\item Welche Auswirkungen hat der Habitatverlust auf die Biodiversität?
	\end{itemize}
\item Kenntnis von gesetzlichen Rahmenbedingungen und Zuständigkeiten für Habitat- und Naturschutz
	\begin{itemize}
	\item Welche gesetzlichen Rahmenbedingungen gibt es in Europa und Deutschland für den Naturschutz?
	\item Wie kann unsere Gesellschaft die Naturhabitate schützen?
	\end{itemize}
\end{itemize}
\end{frame}

%------------------------------------------------

\begin{frame}
\frametitle{Ursachen des Biodiversitätsverlusts}
\centering
\includegraphics[width=1.05\textwidth]{Figures/drivers.jpg}
\end{frame}

%------------------------------------------------

\begin{frame}
\frametitle{\LARGE\alert{Was ist ein Habitat?}}
\onslide<2-> Das Habitat ist ein Lebensraum, den eine Auswahl von Tier- oder Pflanzenarten aus der Lebensgemeinschaft eines Biotops nutzt. Habitate bilden somit Teillebensräume in Biotopen (Campbell und Reece, 2009. Biology)\\
\onslide<1->
\vspace{0.3cm}
\includegraphics[width=\textwidth]{Figures/habitat.jpg}
\end{frame}

%------------------------------------------------

\begin{frame}
\frametitle{Was ist ein Habitat?}
\onslide<1->
\begin{columns}
\column{.4\textwidth}
\alert{Die Rolle des Habitats}\\
\begin{itemize}
\item Territorialität
\item Verbreitung
\item Migration
\item Nahrungsaufnahme
\item Fortpflanzung
\end{itemize}
\hspace{1cm}
\column{.6\textwidth}
\raggedleft
\includegraphics[width=0.9\textwidth]{Figures/hab.png}\\
\centering
\tiny (EEA-FOEN, 2011)
\end{columns}
\end{frame}

%-----------------------------------------------

\begin{frame}
\frametitle{\LARGE\alert{Habitatverlust}}
Habitat wird durch den Menschen anders genutzt (Cain et al. 2011. Ecology)\\
\begin{itemize}
\item Fragmentierung
\item Degradierung
\item Zerstörung
\end{itemize}
\vspace{1cm}
\centering
\includegraphics[width=1.04\textwidth]{Figures/loss.png}\\
\tiny (EEA-FOEN, 2011)
\end{frame}

%------------------------------------------------

\begin{frame}
\frametitle{Habitatverlust}
Sojaplantage im Amazonas\\(Galaz et al. 2018. Nature Eco Evo)\\
\vspace{0.3cm}
\includegraphics[width=\textwidth]{Figures/hab_loss.jpg}
\end{frame}

%------------------------------------------------

\begin{frame}
\frametitle{\LARGE\alert{Fragmentierung}}
Zerschneidung von ehemals zusammenhängenden Lebensräumen (Cain et al. 2011. Ecology)\\
\vspace{0.3cm}
%\includegraphics[width=\textwidth]{Figures/frag.png}\\
\includegraphics[width=\textwidth]{Figures/frag_ex_1.png}
%\centering
%\tiny (EEA-FOEN, 2011)
\tiny (Sinauer Associates, Inc. 2008)
\end{frame}

%------------------------------------------------

%\begin{frame}
%\frametitle{Fragmentierung}
%Amazonasfischgräte (Nepstad et al. 2009. Science)\\
%\vspace{0.3cm}
%\includegraphics[width=\textwidth]{Figures/hab_frag.jpg}
%\end{frame}

%------------------------------------------------

\begin{frame}
\frametitle{\LARGE\alert{Die Prozesse des Habitatverlusts}}
\centering
\includegraphics[width=0.8\textwidth]{Figures/hab_loss_frag_1.jpg}\\
\footnotesize
\begin{itemize}
\item Fragmentierung und Degradierung
\item Tipping Points / Regime Shift (Rocha et al. 2018. Science)
\end{itemize}
\end{frame}

%------------------------------------------------

%\begin{frame}
%\frametitle{Durch Fragmentierung}
%\includegraphics[width=1.05\textwidth]{Figures/frag_ex_1.png}\\
%\centering
%\tiny (Sinauer Associates, Inc. 2008)
%\end{frame}

%------------------------------------------------

%\begin{frame}
%\frametitle{Durch Fragmentierung}
%Landwirtschaft in Bolivien\\(Erickson, 2003. Managing Change)\\[0.5cm]
%\includegraphics[width=1.04\textwidth]{Figures/loss_ex_1.jpg}
%\end{frame}

%------------------------------------------------

\begin{frame}
\frametitle{Tipping Points}
\includegraphics[width=1.04\textwidth]{Figures/tipping.png}\\
\centering
\tiny (Carpenter, 2003. Eco Soc)
\end{frame}

%------------------------------------------------

\begin{frame}
\frametitle{Tipping Points}
Zerstörung der Korallenbank (WWF, 2018)\\[0.5cm]
\includegraphics[width=1.04\textwidth]{Figures/loss_ex_2.png}
\end{frame}

%------------------------------------------------

\begin{frame}
\frametitle{Ursachen des Habitatverlusts}
\begin{columns}
\column{.5\textwidth}
\alert{Durch den Menschen}\\
\begin{itemize}
\item Landwirtschaft
\item Bauaktivitäten
\item Umweltverschmutzung
\item Vielwirtschaft
\item Entwässerung von Feuchtgebieten
\end{itemize}
\hspace{0.5cm}
\column{.5\textwidth}
\raggedleft
\includegraphics[width=0.9\textwidth]{Figures/loss_ex_3.png}\\
\centering
\tiny (Galaz et al. 2018. Nature Eco Evo)
\end{columns}
\end{frame}

%-----------------------------------------------

\begin{frame}
\frametitle{Auswirkungen des Habitatverlusts}
\includegraphics[width=0.74\textwidth]{Figures/impact_1.png}
\includegraphics[width=0.24\textwidth]{Figures/impact_2.png}\\
\hspace{2cm} \tiny (Fahrig, 2003. Ann Rev Eco Evo Sys)\\
\normalsize

\end{frame}

%------------------------------------------------

\begin{frame}
\frametitle{Auswirkungen des Habitatverlusts}
\hspace{-0.3cm} Fragmentierung der Landschaft in Europa (EEA-FOEN, 2011)\\[0.3cm]
\centering
\includegraphics[width=1.04\textwidth]{Figures/euro_frag.png} 
\end{frame}

%------------------------------------------------

\begin{frame}
\frametitle{Auswirkungen des Habitatverlusts}
\centering
\includegraphics[width=1.04\textwidth]{Figures/loss_de.png} 
\end{frame}

%------------------------------------------------

\begin{frame}
\frametitle{Auswirkungen des Habitatverlusts}
Auswirkungen der Straßendichte auf die Abundanz des Feldhasen im Canton Aargau, Schweiz (EEA-FOEN, 2011)\\[0.3cm]
\centering
\includegraphics[width=0.6\textwidth]{Figures/loss_ex_4.png} 
\end{frame}

%------------------------------------------------

\begin{frame}
\frametitle{Habitatschutz: Europa und Deutschland}
Gesetzliche Rahmenbedingungen\\
\centering
\includegraphics[width=1.04\textwidth]{Figures/schutz.png}
\end{frame}

%------------------------------------------------

\begin{frame}
\frametitle{Habitatschutz: Europa und Deutschland}
Gesellschaftliche Herausforderungen\\
\centering
\includegraphics[width=1.04\textwidth]{Figures/schutz_1.jpg}\\
\tiny (Morrison, 2014)
\end{frame}

%------------------------------------------------

\begin{frame}
\frametitle{\LARGE\alert{Aufgabe}}
Bewertet unter Zuhilfenahme der Artikel unterschiedliche Maßnahmen zum Schutz der Habitate in Deutschland.

Die entsprechenden Pdfs zu den Artikeln findet Ihr unter: \alert{https://github.com/AvitBhowmik/testUP})

\end{frame}

%------------------------------------------------

\begin{frame}
\frametitle{}
\centering
\LARGE Vielen Dank für Eure Aufmerksamkeit\\
\normalsize
\begin{columns}
\begin{column}{0.6\textwidth}
\footnotesize
   Email: \href{mailto:avit.bhowmik@futureearth.org}{\alert{avit.bhowmik@futureearth.org}}\\
   Website: \href{http://avitbhowmik.ml/}{\alert{http://avitbhowmik.ml/}}\\
twitter: \href{https://twitter.com/avitbhowmik}{\alert{@avitbhowmik}}\\
   Google Scholar: \href{https://scholar.google.se/citations?user=laRo5pgAAAAJ&hl=en}{\alert{Dr. Avit K. Bhowmik}}
\end{column}
\begin{column}{0.4\textwidth}
    \begin{center}
     \includegraphics[width=0.7\textwidth]{Figures/Ques.png}
     \end{center}
\end{column}
\end{columns}
\end{frame}

%------------------------------------------------

\end{document} 